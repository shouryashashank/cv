\documentclass[10pt, letterpaper]{article}

% Packages:
\usepackage[
    ignoreheadfoot, % set margins without considering header and footer
    top=0.3 cm, % seperation between body and page edge from the top
    bottom=0.0 cm, % seperation between body and page edge from the bottom
    left=0.2 cm, % seperation between body and page edge from the left
    right=0.2 cm, % seperation between body and page edge from the right
    footskip=0.0 cm, % seperation between body and footer
    % showframe % for debugging 
]{geometry} % for adjusting page geometry
\usepackage{titlesec} % for customizing section titles
\usepackage{tabularx} % for making tables with fixed width columns
\usepackage{array} % tabularx requires this
\usepackage[dvipsnames]{xcolor} % for coloring text
\definecolor{primaryColor}{RGB}{0, 0, 0} % define primary color
\usepackage{enumitem} % for customizing lists
\usepackage{fontawesome5} % for using icons
\usepackage{amsmath} % for math
\usepackage[
    pdftitle={John Doe's CV},
    pdfauthor={John Doe},
    pdfcreator={LaTeX with RenderCV},
    colorlinks=true,
    urlcolor=primaryColor
]{hyperref} % for links, metadata and bookmarks
\usepackage[pscoord]{eso-pic} % for floating text on the page
\usepackage{calc} % for calculating lengths
\usepackage{bookmark} % for bookmarks
\usepackage{lastpage} % for getting the total number of pages
\usepackage{changepage} % for one column entries (adjustwidth environment)
\usepackage{paracol} % for two and three column entries
\usepackage{ifthen} % for conditional statements
\usepackage{needspace} % for avoiding page brake right after the section title
\usepackage{iftex} % check if engine is pdflatex, xetex or luatex

% Ensure that generate pdf is machine readable/ATS parsable:
\ifPDFTeX
    \input{glyphtounicode}
    \pdfgentounicode=1
    \usepackage[T1]{fontenc}
    \usepackage[utf8]{inputenc}
    \usepackage{lmodern}
\fi

\usepackage{charter}

% Some settings:
\raggedright
\AtBeginEnvironment{adjustwidth}{\partopsep0pt} % remove space before adjustwidth environment
\pagestyle{empty} % no header or footer
\setcounter{secnumdepth}{0} % no section numbering
\setlength{\parindent}{0pt} % no indentation
\setlength{\topskip}{0pt} % no top skip
\setlength{\columnsep}{0.15cm} % set column seperation
\pagenumbering{gobble} % no page numbering
\setlength{\parskip}{-0.2em}


\titleformat{\section}{\needspace{4\baselineskip}\bfseries\large}{}{0pt}{}[\vspace{1pt}\titlerule]

\titlespacing{\section}{
    % left space:
    -1pt
}{
    % top space:
    0.1 cm
}{
    % bottom space:
    0.1 cm
} % section title spacing

\renewcommand\labelitemi{$\vcenter{\hbox{\small$\bullet$}}$} % custom bullet points
\newenvironment{highlights}{
    \begin{itemize}[
        topsep=0.03 cm,
        parsep=0.02 cm,
        partopsep=0pt,
        itemsep=0pt,
        leftmargin=0 cm + 5pt
    ]
}{
    \end{itemize}
} % new environment for highlights


\newenvironment{highlightsforbulletentries}{
    \begin{itemize}[
        topsep=0.10 cm,
        parsep=0.10 cm,
        partopsep=0pt,
        itemsep=0pt,
        leftmargin=10pt
    ]
}{
    \end{itemize}
} % new environment for highlights for bullet entries

\newenvironment{onecolentry}{
    \begin{adjustwidth}{
        0 cm + 0.00001 cm
    }{
        0 cm + 0.00001 cm
    }
}{
    \end{adjustwidth}
} % new environment for one column entries

\newenvironment{twocolentry}[2][]{
    \onecolentry
    \def\secondColumn{#2}
    \setcolumnwidth{\fill, 4.5 cm}
    \begin{paracol}{2}
}{
    \switchcolumn \raggedleft \secondColumn
    \end{paracol}
    \endonecolentry
} % new environment for two column entries

\newenvironment{threecolentry}[3][]{
    \onecolentry
    \def\thirdColumn{#3}
    \setcolumnwidth{, \fill, 4.5 cm}
    \begin{paracol}{3}
    {\raggedright #2} \switchcolumn
}{
    \switchcolumn \raggedleft \thirdColumn
    \end{paracol}
    \endonecolentry
} % new environment for three column entries

\newenvironment{header}{
    \setlength{\topsep}{0pt}\par\kern\topsep\centering\linespread{1.5}
}{
    \par\kern\topsep
} % new environment for the header

\newcommand{\placelastupdatedtext}{% \placetextbox{<horizontal pos>}{<vertical pos>}{<stuff>}
  \AddToShipoutPictureFG*{% Add <stuff> to current page foreground
    \put(
        \LenToUnit{\paperwidth-2 cm-0 cm+0.05cm},
        \LenToUnit{\paperheight-1.0 cm}
    ){\vtop{{\null}\makebox[0pt][c]{
        \small\color{gray}\textit{Last updated in September 2024}\hspace{\widthof{Last updated in September 2024}}
    }}}%
  }%
}%

% save the original href command in a new command:
\let\hrefWithoutArrow\href

% new command for external links:


\begin{document}
    \newcommand{\AND}{\unskip
        \cleaders\copy\ANDbox\hskip\wd\ANDbox
        \ignorespaces
    }
    \newsavebox\ANDbox
    \sbox\ANDbox{$|$}

    \begin{header}
        \fontsize{25 pt}{25 pt}\selectfont Shourya Shashank

        \vspace{5 pt}

        \normalsize
        \kern 5.0 pt%
        \mbox{\hrefWithoutArrow{mailto:shouryashashank@gmail.com}{shouryashashank@gmail.com}}%
        \kern 5.0 pt%
        \AND%
        \kern 5.0 pt%
        \mbox{\hrefWithoutArrow{tel:+91-7478004777}{+91-7478004777}}%
        \kern 5.0 pt%
        \AND%
        \kern 5.0 pt%
        \mbox{\hrefWithoutArrow{https://github.com/shouryashashank}{github.com/shouryashashank}}%
        \kern 5.0 pt%
        \AND%
        \kern 5.0 pt%
        \mbox{\hrefWithoutArrow{https://www.linkedin.com/in/shourya-shashank/}{linkedin.com/in/shourya-shashank}}%
        \kern 5.0 pt%
    \end{header}

    \vspace{5 pt - 0.3 cm}

    \section{EDUCATION}

        \begin{twocolentry}{
            July 2016 – May 2021
        }
            \textbf{Indian Institute of Technology, Kharagpur}\end{twocolentry}

        \vspace{0.10 cm}
        \begin{onecolentry}
            B.Tech \& M.Tech (Dual Degree) in Mining Engineering \& Safety Engineering, \\
            \vspace{0.10 cm}
            Coursework: Artificial Intelligence: Foundations and Applications, Programming and Data Structures, Quantitative Decision Making
        \end{onecolentry}

    \section{TECHNICAL SKILLS}
    


        
        \begin{onecolentry}
            \textbf{Languages and Libraries:} C\#, Python, C++, Java, SQL, JavaScript, Rust, TensorFlow, PyTorch, Predacons
        \end{onecolentry}

        \vspace{0.1 cm}

        \begin{onecolentry}
            \textbf{Technologies:} Azure, Docker, Google vertex, Azure AI Studio, Function App, Kafka, DataBricks, Transfoermers\end{onecolentry}

        

    \section{EXPERIENCE}

    
        \begin{onecolentry}
            \textbf{\large Honeywell} \hfill \textit{June 2021 – Present} \\  
            \textbf{\normalsize Advanced Software Development Engineer} \hfill \textit{Bengaluru, Karnataka} \\[0.2cm] % Adds slight spacing
            \textbf{Production Intelligence} \\  
            \begin{adjustwidth}{0.4cm}{0cm}
                \begin{highlights}
                    \item Spearheaded development of a large-scale, cloud-native industrial software, ensuring robust, scalable and efficient architecture.
                    \item Achieved 99.9\% application availability by utilizing durable functions and redundant worker nodes, ensuring minimal downtime.
                    \item Designed and developed a Python expression evaluator, enabling users to write and execute custom calculation formulas seamlessly.
                    \item Increased data backfill speed by 15x through parallelization and bulk data insertion techniques, significantly improving performance.
                    \item Managed Azure infra, including function apps, Kubernetes, file shares, and service bus, ensuring seamless integration and operation.
                    \item Improved reliability and speed of inter-microservice calls by optimizing communication protocols and reducing latency.
                    \item Effectively utilized Druid for time-series data management, ensuring efficient data storage and retrieval for analysis and prediction.
                    \item Designed a system to run scheduled calculations on KPIs, providing timely insights and performance metrics for end-users.
                    \item Delivered tailored solutions for key industries such as mining, and manufacturing, addressing their unique data management needs.
                    \item Leveraged reason analysis and other advanced techniques to generate accurate predictions, aiding in proactive decision-making.
                    
                \end{highlights}
            \end{adjustwidth}

            \vspace{0.2 cm}
            \textbf{Honeywell Forge AI} \\  
            \begin{adjustwidth}{0.4cm}{0cm}
                \begin{highlights}
                    \item Accomplished the design and development of a RAG-based Agentic AI tool that effectively communicates across internal GET APIs, databases (sql, time-series), and logbooks, facilitating seamless data exchange and operational efficiency.
                    \item Implemented an AI tool that answers queries through a user-friendly chat interface, improving the speed of information retrieval.
                    \item Enhanced operational efficiency and user interaction by enabling smooth operations on KPIs and assets through agentic AI.
                    \item Established Forge AI infrastructure by integrating Azure AI across Honeywell's ecosystem, enhancing the company's AI capabilities.
                    \item Developed a GPT-4o based migration tool for legacy application calculations, eliminating manual migration efforts and Human errors.
                    \item Migrated user code and configurations from VBScript to Python, transitioning configurations to the latest cloud infrastructure.
                    \item Achieved savings of over 3000 days of manual work through automated migration with AI integration, accelerating project timelines.
                    
                \end{highlights}
            \end{adjustwidth}

            \vspace{0.2 cm}
            \textbf{\normalsize Software Engineer Intern} \hfill \textit{June 2020 -  August 2020} \\[0.2cm] % Adds slight spacing
            \begin{adjustwidth}{0.4cm}{0cm}
                \begin{highlights}
                    \item Successfully migrated Process Safety Analyser code base from DotNet framework to DotNet core, achieving platform independence.
                    \item Developed multiple Class Libraries, enhancing secure handling and analysis of SQL database from sensors Data and log files.
                    \item Contributed to a pilot project, which led to Honeywell's decision to migrate entire Connected Industrial systems to the cloud.
                    \item Ensured optimized performance and scalability, as measured by reduced processing time and improved reliability due to redundancy.
                \end{highlights}
            \end{adjustwidth}

            \vspace{0.2 cm}
            \textbf{\large  Quest Global} \hfill \textit{May 2019 – July 2019} \\  
            \textbf{Deep Learning Internship} \\  
            \begin{adjustwidth}{0.4cm}{0cm}
                \begin{highlights}
                    \item Developed monocular depth estimation to avoid collisions, utilizing an Encoder-Decoder trained on the KITTI dataset.
                    \item Generated depth maps by integrating image and LIDAR data through a DenseNet201 based Encoder-Decoder model.
                    \item Achieved object detection using SSD MobileNet V2, with deployment on NVIDIA TX1, ensuring 30ms response time and 15 FPS.
                    \item Completed a 2-month intensive internship focused on enhancing depth estimation and object detection technologies.
                \end{highlights}
            \end{adjustwidth}

            \vspace{0.2 cm}
            \textbf{\large  Blueseed Ventures} \hfill \textit{December 2018} \\  
            \begin{adjustwidth}{0.4cm}{0cm}
                \begin{highlights}
                    \item Developed a backend and database management system using Node.js and MongoDB for web, Android, and iOS applications.
                    \item Designed a user-friendly UX for a registration system, which enables the accurate estimation of power usage for landed properties. 
                    \item Successfully developed and implemented a community feed system using Google Firebase, designed for testing and enhancements.
                \end{highlights}
            \end{adjustwidth}
            
        \end{onecolentry}

        \vspace{0.2 cm}

        \section{PROJECTS}



        
    \begin{onecolentry}{
    }
        \textbf{PREDACONS | Founder and Maintainer} \hfill \textit{October 2023 - Present}\end{onecolentry}

    \vspace{0.10 cm}
    \begin{onecolentry}
        \begin{highlights}
            \item Established Predacons, a versatile Python library for simplified and flexible training of large language models (LLMs).
            \item Developed user-friendly functions for seamless data handling and model training, enhancing usability and flexibility.
            \item Automated optimization to dynamically adjust fine-tuning, accommodating VRAM limits while allowing user-defined configurations.
            \item Created Predacons Server, an OpenAI-compatible web server, enabling effortless hosting of any LLM model.
            \item Introduced Predacons Agents,that enhance LLMs with data analysis, RAG, web scraping, decision-making, and code interpretation.
            \item Boosted output generation speed by up to 6x, significantly improving performance over standard methods.
            \item Integrated comprehensive features from data preprocessing to real-time chat generation, streamlining the NLP workflow.
            \item Lowered barriers to entry for complex NLP tasks, empowering developers, researchers, and enthusiasts to innovate and excel.
        \end{highlights}
    \end{onecolentry}
    \vspace{0.10 cm}
    
    
\end{document}